\documentclass[12pt,fleqn]{article}
\usepackage{xiiiemc}
\usepackage{natbib}
\usepackage{fancyhdr}
\usepackage{color}
\usepackage{wallpaper} 
\usepackage{titlesec}   %% Define space between paragraph e section
\usepackage{float} 	%% Use to fix Figure or Table: ex: \begin{table}[H]
\usepackage[usenames,dvipsnames]{xcolor}
\usepackage{hyperref} 	%% Use to fix Figure or Table: ex: \begin{table}[H]
\hypersetup{
	%pagebackref=true,
	pdfcreator={LaTeX with abnTeX2},
	pdfkeywords={abnt}{latex}{abntex}{USPSC}{trabalho acadêmico}, 
	colorlinks=true,       		% false: boxed links; true: colored links
	linkcolor=blue,          	% color of internal links
	citecolor=blue,        		% color of links to bibliography
	filecolor=magenta,      		% color of file links
	urlcolor=blue,
	allbordercolors=black,
	bookmarksdepth=4
}
\usepackage{tocloft}
\titleformat{\section}
  {\normalfont\bfseries}{\thesection.}{0.5em}{}
\renewcommand\cftsecaftersnum{.} 
\renewcommand\thesection{\arabic{section}}
\renewcommand\thesubsection{\thesection.\arabic{subsection}}



%%%%Don't edit this block. It reduces the spacing between the lines of the references
\let\OLDthebibliography\thebibliography
\renewcommand\thebibliography[1]{\OLDthebibliography{#1} \setlength{\parskip}{0pt}\setlength{\itemsep}{0pt plus 0.3ex}}

%%-----------------------------------------------EDIT-----------------------------------------------
\title{A SIMPLE MODEL FOR THE LITERARY ANALYSIS BY TEXT ANALYTICS: COMPARATIVE APPROACH AND EXAMPLE ON JAMES JOYCE AGAINST SHAKESPEARE AND THE BIBLE} 
% Mining James Joyce against reference literature: differences, similarities and the ubiquity of the reader

%%-----------------------------------------------EDIT----------------------------------------------
\author
    {\rm \begin{tabular}{l} 
    \textbf{Renato Fabbri}$^{1}$ - {\textnormal renato.fabbri@gmail.com}\\%
    \textbf{Luis Henrique Garcia Ferreira}$^{2}$ - {\textnormal henrique@hotmail.com}\\
    {\fontsize{11}{0}\selectfont $^{1}$University of São Paulo, Institute of Mathematical and Computer Sciences - São Carlos, SP, Brazil}\vspace*{-0.05cm} \\
    {\fontsize{11}{0}\selectfont $^{2}$University of Campinas, Institute of Language Studies - Campinas, SP, Brazil}\vspace*{-0.05cm}\\
  \end{tabular}}
%%----------------------------------------------------------------------------------------------

\fancypagestyle{firspagetstyle}
{
	\lhead{}
	\fancyhead[C]{%
		\includegraphics[width=0.9\linewidth]{logo}\\%
		{\scriptsize \fontfamily{phv}\fontseries{b}\selectfont \color[rgb]{0.45,0.45,0.45}
		16 a 19 de Outubro de 2017\\
		Instituto Politécnico - Universidade do Estado de Rio de Janeiro\\
		Nova Friburgo - RJ\\
	    }
	}
	\renewcommand{\headrulewidth}{0.0pt}
	\fancyfoot[C]{\footnotesize \parbox{15cm} {\centering  \fontsize{7.5}{0}\selectfont \it Anais do XX ENMC – Encontro Nacional de Modelagem Computacional e VIII ECTM – Encontro de Ciências e Tecnologia de Materiais,  Nova Friburgo, RJ – 16 a 19 Outubro 2017}} % \ttfamil
	\rhead{}
}


\begin{document}
\maketitle

\thispagestyle{firspagetstyle}

\fancyhead[L]{\footnotesize{\fontsize{7.5}{0}\selectfont \it XX ENMC e VIII ECTM\\
	16 a 19 de Outubro de 2017\\
	Instituto Politécnico Universidade do Estado do Rio de Janeiro – Nova Friburgo - RJ\\}}
\renewcommand{\headrulewidth}{0.0pt}
\fancyfoot[C]{\footnotesize \parbox{15cm} {\centering  \fontsize{7.5}{0}\selectfont \it Anais do XX ENMC – Encontro Nacional de Modelagem Computacional e VIII ECTM – Encontro de Ciências e Tecnologia de Materiais,  Nova Friburgo, RJ – 16 a 19 Outubro 2017}} % \ttfamil
\rhead{}

\begin{abstract}
    Literary analysis, criticism or studies is a largely valued field with dedicated journals and researchers
    which remains mostly within the humanities scope.
    Text analytics is the computer-aided process of deriving information from texts.
    In this article we describe a simple and generic model for performing literary analysis
    using text analytics.
    The method relies on statistical measures of: 1) token and sentence sizes and
    2) Wordnet synset features.
    These measures are then used in Principal Component Analysis where the texts to be analyzed
    are observed against Shakespeare and the Bible, regarded as reference literature.
    The model is validated by analyzing selected works from James Joyce (1882-1941),
    one of the most important writers of the 20th century.
    We argue for the consistency of this approach, the reasons why we did not use other
    techniques (e.g. part-of-speech tagging) and ways by which the analysis might be adapted and enhanced.
\end{abstract}

\keywords{\em{Text analytics, Literary criticism, Wordnet, Shakespeare, Bible}}

\pagestyle{fancy}

\section{INTRODUCTION}
Literary criticism is performed by intellectuals using various techniques,
including intuition and contextualization through erudition~\citep{litCri}.
Text analytics is usually considered a synonym of text mining,
i.e. data mining applied to textual data, the extraction of meaningful information
from texts by means of computer-aided analysis.
A difference can be established nevertheless:
text mining is more associated to earlier applications (e.g. dating to the 1980s)
and to specific tasks, while the term
text analytics is more frequent nowadays and might be related to a
less purposeful processing of textual data.
Accordingly, for example, a word cloud is more easily associated to
text analytics while a search engine is more promptly associated to text mining~\citep{tmWiki}.

In this work we propose a very simple and generic model for literature analysis
by means of statistical measures, Principal Component Analysis and comparison against
reference literature.
The uncomplicated methods favor the collaboration between researchers of different background.
For example: a computer science professional can understand, adapt and expand the techniques
while a literature scholar can deepen the interpretation and the relevance of the conclusions.

Section~\ref{sec:matMet} describes the corpus and methods.
Section~\ref{sec:res} is dedicated to presentation and discussion of results.
Section~\ref{sec:conc} holds conclusions and further work considerations.

\section{MATERIALS AND METHODS}\label{sec:matMet}
\subsection{Corpus}
This work encompasses a comparison of the literature to be analyzed against
reference literature.
What is regarded as reference literature is arbitrary and we chose them,
within this presentation and first formalization, to be the two utmost references
of the English language~\citep{bib,shake}:
\begin{itemize}
    \item the complete works from William Shakespeare as given by the
        publication in the Gutenberg Project~\citep{shakWhole}:
        36 plays (tragedies, comedies and historical) and poetry (2 batches).
        Shakespeare is often recognized as the greatest writer of the English language
        and is a universal reference of literature.
    \item The 80 books of the King James Bible, including Old Testament (39 books),
        Apocrypha (14 books) and New Testament (27 books).
        This is the most referenced English translation of the bible.
        These books are also universally accredited for their influence in English literature.
\end{itemize}

We should emphasize that changing this reference literature is very straightforward.
One should only provide the corresponding text files and modify the scripts to read the
intended records.
If the works are well-known, the process should require only a quick search on the web
(e.g. within Gutenberg or Archive.org projects), saving the text locally and then changing
filenames in the scripts.
Some possibilities include: other masters of English literature;
a selection of poets; works from scientific literature of a specific field;
works in other language, such as Machado de Assis and Clarice Linspector
if analyzing works in Brazilian Portuguese.
There is no reason why the corpus should not include data streaming or access to online resources,
such as Wikipedia pages.

To illustrate and validate the method,
we performed and herein report an analysis of a selection of works written by James Joyce:
\begin{itemize}
    \item Stephen Hero: written around 1905 and published posthumously in 1944, an autobiographical novel of which part is lost
        (Joyce threw it on fire after a number of rejections by publishers).
    \item Dubliners: published in 1914, it is a collection of 15 short stories about Dublin's middle class.
    \item A Portrait of the Artist as a Young Man: published in 1916, a condensed and reworked version of Stephen Hero. 
    \item Ulysses: published in 1922, considered one of the most important works of the modernist literature.
    \item Finnegans Wake: published in 1939, often considered one of the most difficult fictional works of the English language.
\end{itemize}

\subsection{Pre-processing}
The reference literature (Shakespeare and Bible books) were separated and cleaned into
individual files.
As both collections do not hold well defined paragraph structures,
these were discarded.
These routines can be inspected through reading the scripts in Table~\ref{tab:files}.

\begin{table}[H] % !htbp 
	\caption{Files related to the analysis method proposed in this article.
	All files are found in a public git repository~\citep{repo}.}\label{tab:files}
\vspace{12pt}
\centering{}
	\begin{tabular}{  c | p{8cm} }
	\textbf{File}           & \textbf{Description} \\\hline
	\texttt{scripts/analysis.py}  & Python script that makes the initial quantification of the books \\
	\texttt{scripts/analysis2.py}  & Python script that performs PCA and renders the figures with scatter plots \\
	\texttt{scripts/bibleSeparation.py}  & Python scripts that separates the King James Bible text into files with individual books \\
	\texttt{scripts/shakespeareSeparation.py}  & Python scripts that separates the text with the complete works of William Shakespeare into files with individual books \\
        \texttt{corpus/*}  & text files corresponding to individual books from Shakespeare, Joyce and the bible  \\
	\texttt{latex/*}  & the PDF of this article and the files necessary to render it. It is the main documentation of the proposed analysis. \\
\end{tabular}
\end{table}


\subsection{Analysis routine}
As modeled until the moment, the analysis is performed by: the achievement of meaningful sets of textual elements,
quantifying their incidences, taking overall measurements of these quantifications in each of the books,  
performing Principal Component Analysis (PCA) of the books in the measurements space, plotting the books within principal components
and most meaningful measures, interpreting the results.
We should look at each of these phases:
\begin{itemize}
    \item Achievement of meaningful sets of textual elements: the original texts were separated into sets of: sentences, tokens, stopwords, known words (which are not stopwords), punctuations, tokens which are not stopwords or punctuation, Wordnet synsets of each known word.
    \item Quantization: each of the sets above were quantified by: means of their sizes in number of characters of each element, or by means of the number of elements they contain, or by means of synset characteristics (only depth was used in the example analysis).
    \item For the PCA, all the books were considered together. The z-score of each dimension (measure type) was performed to avoid meaningless prevalence of some measures over others (z-score of measures $x_i$ is $x'_i = \frac{x_i-\mu(x)}{\sigma(x)}$ where $\mu(x)$ is the mean of all $x_i$ and $\sigma(x)$ is the standard deviation). Then PCA was performed as usual: performing the eigendecomposition of the covariance matrix (where entry $m_{ij}$ if the covariance of measures $i$ with measures $j$) and observing the eigenvectors associated to the greatest eigenvalues.
    \item For visual inspection of the resulting structures, we used scatter plots of principal components and of measures which were relevant to our analysis of James Joyce's works.
    \item The interpretation of the results, we made discussions about literary analysis and the analysis of James Joyce before performing the quantitative analysis described above.
        When the final figures and measures were done, we had another round of considerations about what they revealed.
\end{itemize}

We discarded using other techniques mainly because of three reasons:
1) other methods involve greater complexity and would not favor the communication between interested parties;
2) e.g. part-of-speech tagging relies heavily on the vocabulary and the syntactic structure
which are used with deep innovations by literary authors, especially from the last century and thereon;
3) using only the measures mentioned above, we already reached 20 dimensions.
Nevertheless, we encourage adapting the method by inclusion of other measures and of other analysis procedures beyond PCA,
and we will probably do so in further considerations of this efforts.

\section{RESULTS AND DISCUSSION}\label{sec:res}

\begin{figure}[!htbp] %h or !htbp
\vspace{-2pt}
\begin{center}
\includegraphics[height=6.7cm,width=9cm]{figs/pca}%
\caption{First two principal components.}
\label{fig:pca1}%
\end{center}
\end{figure}

Figures~\ref{fig:pca1},~\ref{fig:pca2} and~\ref{fig:pca3} exposes the works of Joyce,
Shakespeare and bible books within the principal components.
As the second and third component held very near spreads,
and the first two components summed only $\approx 50\%$ of all dispersion,
we chose to use the first three components.
As can be noticed, Joyce's works are very distinct from Shakespeare,
and some of them are also very distinct from bible books.
Even so, some of them fall near bible books.

\begin{figure}[!htbp] %h or !htbp
\vspace{-2pt}
\begin{center}
\includegraphics[height=6.7cm,width=9cm]{figs/pca2}%
\caption{First and third principal components.}
\label{fig:pca2}%
\end{center}
\end{figure}


Figures~\ref{fig:abst1} and~\ref{fig:abst2} are direct plots of measures
we idealized to probe the extent of the need of interpretation by the reader.
The first of these plots is dedicated to the amount of unknown words,
and the conclusion is that some of the works have an very distinctive amount
of unknown words, but all of them fall on the greater amount of unknown words
among the most meaningful tokens when the same amount of unknown words is considered.
The second plot is dedicated to synset depth.
The lower the depth, the more abstract the concept is.
In this plot, we conclude that three of the works by Joyce lie on
the more abstract margin among the reference works,
but two of them lie within the middle and the more concrete (less abstract)
books.

\begin{figure}[!htbp] %h or !htbp
\vspace{-2pt}
\begin{center}
\includegraphics[height=6.7cm,width=9cm]{figs/pca3}%
\caption{Second and third principal components.}
\label{fig:pca3}%
\end{center}
\end{figure}

We propose to validate and illustrate the method by analyzing
works by Joyce, but, as this is the first work of the kind which analyzes
Shakespeare and the bible, as far as the authors know,
some considerations about them are also opportune.
First, the works by Shakespeare lie in a notably more restricted domain
when compared against the bible.
Second, they are perfectly distinguishable with respect to the first two
principal components: a simple Bayesian inference of neural network
should be able to correctly classify a book from one group or the other.
Third, Shakespeare uses a less abstract language at least in the sense captured
by the depth of the synsets.
This diversity is convenient for a reference literature to compare something against.

\begin{figure}[!htbp] %h or !htbp
\vspace{-2pt}
\begin{center}
\includegraphics[height=6.7cm,width=9cm]{figs/abst1}%
\caption{Synset depths. Lower depth is regarded here as evidence of abstraction.}
\label{fig:abst1}%
\end{center}
\end{figure}

Finally,
we believe to have reached a good result in terms of the
model proposed for the analysis.
The analysis is very simple, which favors both elaboration of
variants and the understanding by interested researchers which are potentially from
diverse and multidisciplinary background.
It is robust, in the sense that it does not rely in canonical vocabulary or syntactic structures.
Furthermore, the method is very fast:
pre-processing and then processing and rendering the figures can all be performed in dozens of seconds.

\begin{figure}[!htbp] %h or !htbp
\vspace{-2pt}
\begin{center}
\includegraphics[height=6.7cm,width=9cm]{figs/abst2}%
\caption{Fraction of known words among all tokens and significant words.
    Lower faction is regarded here as evidence of abstraction.}
\label{fig:abst2}%
\end{center}
\end{figure}

% \newpage %
\section{CONCLUSIONS AND FUTURE WORK}\label{sec:conc}
The analysis method proposed yields interesting results for literary criticism.
It is robust, easily adaptable and fast.
Also, the online availability of the scripts and the reference corpus,
all in public domain, facilitates reuse and the achievement of derivatives.
The example analysis,
revealed both distinctive traces of the set and can be used to argue quantitatively
in favor of the thesis that the style of Joyce calls the reader to fill the
meaning gaps generated by the abstraction.

In further efforts, we should:
\begin{itemize}
    \item Deepen the analysis of the reference literature (book by Shakespeare and in the bible)
        to better contextualize any literature we consider against them.
    \item Expand the use of Wordnet to encompass synonymy, antonymy, meronymy, etc.
        Also to consider specific roots of nouns, adjectives, verbs and adverbs.
    \item Report this endeavor to the literary criticism scholars.
        This should be done at least in two ways: by describing the method and its relevance
        within the humanities background;
        and by exposing results from analyzing specific authors, such as Joyce, Ezra Pound, etc.
    \item Consider other measures of abstraction.
        Can we use length of words and sentences?
    \item Vary the methods and state reasonable generic bounds e.g. for splitting a work to
        obtain more data points.
\end{itemize}

\subsection*{\textit{Acknowledgements}}
The authors thank the open source software developers,
especially those who enabled this work by developing
the Python language and the NLTK;
the open culture movement, especially the collaborators
of the Gutenberg and the Archive.org projects, which
enabled this work by making the literary works available;
the IFSC/USP, ICMC/USP and IEL/UNICAMP researchers for their attentive
collaboration whenever we required opinions and directions for learning and researching.




% ------------------------------------------------------------------------
\begin{thebibliography}{99}
\fontsize{11}{0}\selectfont
\bibitem[Richards, 2003]{litCri}
Richards, I. A. (2003). Principles of literary criticism. Routledge.

\bibitem[Wikipedia, 2017]{tmWiki}
    Text mining. (2017, July 31). In Wikipedia, The Free Encyclopedia. Retrieved 02:02, August 10, 2017, from \url{https://en.wikipedia.org/w/index.php?title=Text_mining&oldid=793168242}

\bibitem[Norton, 2000]{bib}
Norton, D. (2000). A history of the English Bible as literature. Cambridge University Press.

\bibitem[Boom, 1999]{shake}
Bloom, H. (1998). Shakespeare: The invention of the human. Riverhead Trade (Paperbacks).

\bibitem[Shakespeare, 1994]{shakWhole}
Shakespeare, W. (1994). The Complete Works of William Shakespeare. The Project Gutenberg.

\bibitem[Fabbri \& Ferreira, 2017]{repo}
Fabbri, R., Ferreira, L.H.G. (2017). Public git repository with the scripts and corpora for the quantitative analysis to support literary criticism. From \url{https://github.com/ttm/joyce}

\end{thebibliography}
% ------------------------------------------------------------------------

%For papers written in Portuguese or Spanish.

%\begin{center}
%  TITLE IN ENGLISH
%\end{center}

%\def\abstractname{Abstract}%

%\begin{abstract}
%   Abstract in english
%\end{abstract}

%\keywords{\em{Keywords in english}}

\end{document}
